\documentclass{article}
\usepackage[italian]{babel}
\usepackage[a4paper,top=2cm,bottom=2cm,left=3cm,right=3cm,marginparwidth=1.75cm]{geometry}

% Packages
\usepackage{amsmath}
\usepackage{tikz}
\usepackage{caption}
\usepackage{algorithm}
\usepackage{algpseudocode}
%\usepackage{graphicx}
%\usepackage[colorlinks=true, allcolors=blue]{hyperref}

\tikzset{
  treenode/.style = {align=center, circle, black, inner sep=0pt, text centered,
    font=\sffamily, draw=black, text width=1.5em, very thick}
}

\title{Struttura Dati Succinta per la Gestione di Sequenze di Parentesi ben Bilanciate}
\author{Fabrizio Brioni}

\begin{document}
\maketitle

\begin{abstract}
In questo documento verrà mostrata una struttura succinta per memorizzare ed effettuare operazioni su una sequenza di parentesi ben bilanciata. Inizialmente verrà descritta la struttura dati segment tree, dopodiché vi sarà la descrizione di un'organizzazione a due livelli (simile alla struttura di jacobson per il rank) per la memorizzazione di informazioni riguardanti le sequenze di parentesi ben bilanciate e infine una spiegazione di come integrare le due strutture per gestire operazioni di tipo find\_close, find\_open, find\_enclose in modo efficiente e succinto.
\end{abstract}

\section{Introduzione}
Una sequenza $V$ di parentesi è definita ben bilanciata se in ciascun prefisso di $V$ il numero di parentesi aperte è maggiore o uguale al numero di parentesi chiuse e il numero complessivo di parentesi aperte in $V$ è uguale al numero di parentesi chiuse. In termini meno formali, può essere definita come una sequenza di parentesi da cui è possibile ottenere un'operazione aritmetica valida inserendo eventuali numeri e operazioni. Alcune operazioni utili eseguibili su tali sequenze sono le sequenti:
    \begin{itemize}
    \item \texttt{find\_close(i)}: data una parentesi aperta in posizione $i$, calcolare la posizione delle corrispettiva parentesi chiusa;
    \item \texttt{find\_open(i)}: data una parentesi chiusa in posizione $i$, calcolare la posizione delle corrispettiva parentesi aperta;
    \item \texttt{find\_enclose(i)}: data una parentesi in posizione $i$, calcolare le posizioni $(x,y)$ della coppia di parentesi corrispondenti che la racchiudono. Nel caso vi siano più coppie con tale proprietà, cercare quella più interna, ovvera quella che non contiene altre coppie valide (per una definizione più formale si legga la sezione \textbf{Notazioni}). 
    \end{itemize}
In questo documento verrà mostrato come, data una sequenza di parentesi ben bilanciate di lunghezza $2n$, eseguire tali operazioni con una complessità temporale di $\mathcal{O}(\log{n})$ utilizzando $2n + o(n)$ bits di memoria.

\section{Notazioni}
Sia $v$ una sequenza di parentesi di lunghezza $2n$ numerate da $0$ a $2n-1$ e sia $v_i$ la parentesi di indice $i$. Per comodità supponiamo che un valore $v_i=1$ indichi una parentesi aperta in posizione $i$ e un valore $v_i=0$ indichi una parentesi chiusa in posizione $i$. Definiamo l'eccesso di una posizione $i$ come:
    $$
    \text{excess}_v(i) = |\{j :\ j<i \land v_i=1\}|-|\{j :\ j\leq i \land v_i=0\}|
    $$
ovvero la differenza tra il numero di parentesi aperte che precedono $i$ (escluso) e il numero di parentesi chiuse che precedono $i$ (questa volta incluso).
Ne segue che una sequenza di parentesi è ben bilanciata se e solo se:
    \begin{align*}
    \text{excess}_v(i) &\geq 0 \qquad 0\leq i < 2n-1 \\
    \text{excess}_v(2n-1) &= 0
    \end{align*}
Inoltre ne consegue la seguente definizione delle operazioni di $\text{find\_close},\text{find\_open}$ e $\text{find\_enclose}$:
    \begin{align*}
    \text{find\_close}_v(i) &= \min\{j :\ j>i \land \text{excess}_v(j)=\text{excess}_v(i)\} \\
    \text{find\_open}_v(i) &= \max\{j :\ j<i \land \text{excess}_v(j)=\text{excess}_v(i)\} \\
    \text{left\_enclose}_v(i) &= \max\{j :\ j<i \land \text{excess}_v(j)+1=\text{excess}_v(i)\} \\
    \text{right\_enclose}_v(i) &= \text{find\_close}_v(\text{left\_enclose}_v(i)) = \min\{j :\ j>i \land \text{excess}_v(j)+1=\text{excess}_v(i)\} \\
    \text{find\_enclose}_v(i) &= (\text{left\_enclose}_v(i),\text{right\_enclose}_v(i))
    \end{align*}
Inoltre useremo $t^v$ per indicare una sequenza di interi di lunghezza $2n$ tale che:
    $$
    t^v_i=\text{excess}_v(i) \qquad 0\leq i < 2n-1
    $$
Infine definiamo una funzione $\texttt{find\_succ}(x,i,v)$ che data una sequenza $x_0,x_1,\dots$ di interi e due interi $i$ e $v$ restituisca l'indice del primo elemento che segue $x_i$ ed ha valore minore o uguale a $v$:
    $$
    \text{find\_succ}(x,i,v)=\min\{j :\ j>i \land x_j\leq v\}
    $$
in modo analogo definiamo una funzione $\texttt{find\_prev}(x,i,v)$ che restituisca l'indice dell'ultimo elemento che precede $x_i$ ed ha valore minore o uguale a $v$:
    $$
    \text{find\_prev}(x,i,v)=\max\{j :\ j<i \land x_j\leq v\}
    $$

\section{Segment Tree}
Il Segment Tree relativo a una sequenza $x_0,x_1,\dots,x_{n-1}$ di $n$ valori è un albero binario che ha come radice un nodo contenente informazioni relative all'intera sequenza (come ad esempio la somma, il valore massimo o il minimo) e (se $n\neq 1$) avente come sottoalbero sinistro un Segment Tree relativo alla sequenza $x_0,\dots,x_{\left\lfloor{\frac{n-1}{2}}\right\rfloor}$ e come sottoalbero destro un SegmentTree relativo alla sequenza $x_{\left\lfloor{\frac{n-1}{2}}\right\rfloor+1},\dots,x_{n-1}$. Dato un nodo $k$ di tale albero, con $k.data$ indicheremo l'informazione contenuta (nel nostro caso sarà sempre contenuto il minimo dell'intervallo di competenza) e con $k.left$ e $k.right$ rispettivamente il figlio sinistro e destro.
    \begin{center}
        \begin{tikzpicture}[level/.style={sibling distance = 5cm/#1,
          level distance = 1.5cm}] 
        \node [treenode] {3}
            child{ 
                node [treenode] {3} 
                child{ 
                    node [treenode] {5} 
                	child{ 
                	    node [treenode] {5} 
                	}
        			child{
        			    node [treenode] {7}
        			}
                }
                child{ 
                    node [treenode] {3}
        			child{ 
        			    node [treenode] {18}
        			}
        			child{
        			    node [treenode] {3}
        			}
                }                            
            }
            child{ 
                node [treenode] {6}
                child{ 
                    node [treenode] {6} 
        			child{ 
        			    node [treenode] {16}
        			}
        			child{
        			    node [treenode] {6}
        			}
                }
                child{ 
                    node [treenode] {15}
                }
        	}
        ;
        \end{tikzpicture}
        \captionsetup{type=figure, width=.76\linewidth}
        \captionof{figure}{Segment Tree relativo alla sequenza $5,7,18,3,16,6,15$. In ciascun nodo è contenuto il valore minimo della sequenza di cui si occupa.}
    \end{center}

\subsection{Costruzione}
La seguente procedura restituisce la radice del Segment Tree relativo alla sequenza $x$ di lunghezza $n=size(x)$:
    \begin{algorithmic}[1]
    \Procedure{Build}{$x$}
        \State $root\gets$\Call{Recursive\_build}{$x,0,size(x)-1$}
        \State \textbf{return} $root$
    \EndProcedure
    \State
    \Function{Recursive\_build}{$x,l,r$}
        \State $tmp\gets\text{new Segment Tree node}$
        \If{$l=r$}
            \State $tmp.data\gets x_l$
            \State $tmp.left\gets null$
            \State $tmp.right\gets null$
        \Else
            \State $tmp.left\gets$\Call{Recursive\_build}{$x,l,\left\lfloor{\frac{l+r}{2}}\right\rfloor$}
            \State $tmp.right\gets$\Call{Recursive\_build}{$x,\left\lfloor{\frac{l+r}{2}}\right\rfloor+1,r$}
            \State $tmp.data\gets\min\{tmp.left.data,tmp.right.data\}$
        \EndIf
        \State \textbf{return} $tmp$
    \EndFunction
    \end{algorithmic}
Per una sequenza lunga $n$ il numero $f(n)$ di nodi  creati è:
    \begin{align*}
        f(n)=
        \begin{cases}
            1 \qquad &n=1 \\
            1+f(\left\lfloor{\frac{n}{2}}\right\rfloor)+f(\left\lceil{\frac{n}{2}}\right\rceil) \qquad &n>1
        \end{cases}
    \end{align*}
ne consegue che $f(n)=\mathcal{O}(n)$ e poiché la funzione \texttt{Recursive\_build} viene chiamata una volta per ciascun nodo, la complessità della procedura di costruzione è $\mathcal{O}(n)$. Inoltre supponendo che ciascun elemento delle sequenza abbia un valore compreso fra $0$ e $A$, il numero di bit necessari per contenere le informazioni di tutti i nodi è $\mathcal{O}(n\log{A})$.

\subsection{find\_succ e find\_prev}
Grazie a tale struttura è possibile implementare in modo efficiente la funzione $\text{find\_succ}(x,i,v)$ una volta costruito il Segment Tree relativo ad $x$ la cui radice è \texttt{root}:
    \begin{algorithmic}[1]
    \Procedure{Find\_succ}{$root,x,i,v$}
        \State \textbf{return} \Call{Find\_succ\_recursive}{$root,i,v,0,size(x)-1$}
    \EndProcedure
    \State
    \Function{Find\_succ\_recursive}{$node,ind,val,l,r$}
        \If{$ind\geq r \lor node.data>val$}
            \State $res\gets\infty$
        \ElsIf{$l=r$}
            \State $res\gets l$
        \Else
            \State $res\gets$\Call{Find\_succ\_recursive}{$node.left,ind,val,l,\left\lfloor{\frac{l+r}{2}}\right\rfloor$}
            \If{$res\neq \infty$}
                \State $res\gets$\Call{Find\_succ\_recursive}{$node.right,ind,val,\left\lfloor{\frac{l+r}{2}}\right\rfloor+1,r$}
            \EndIf
        \EndIf
        \State \textbf{return} $res$
    \EndFunction
    \end{algorithmic}
Il numero di nodi visitati e la complessità di tale procedura è $\mathcal{O}(\log{n})$. In modo del tutto analogo è possibile implementare la funzione $\text{find\_prev}(x,i,v)$:
    \begin{algorithmic}[1]
    \Procedure{Find\_prev}{$root,x,i,v$}
        \State \textbf{return} \Call{Find\_prev\_recursive}{$root,i,v,0,size(x)-1$}
    \EndProcedure
    \State
    \Function{Find\_prev\_recursive}{$node,ind,val,l,r$}
        \If{$ind\leq l \lor node.data>val$}
            \State $res\gets -\infty$
        \ElsIf{$l=r$}
            \State $res\gets l$
        \Else
            \State $res\gets$\Call{Find\_prev\_recursive}{$node.right,ind,val,\left\lfloor{\frac{l+r}{2}}\right\rfloor+1,r$}
            \If{$res\neq -\infty$}
                \State $res\gets$\Call{Find\_prev\_recursive}{$node.left,ind,val,l,\left\lfloor{\frac{l+r}{2}}\right\rfloor$}
            \EndIf
        \EndIf
        \State \textbf{return} $res$
    \EndFunction
    \end{algorithmic}

\section{Organizzazione a due livelli delle informazioni}
Data una sequenza $v$ di parentesi ben bilanciate di lunghezza $2n$, la si suddivida in \textit{super blocchi} di dimensione $\log^2{2n}$ (primo livello) e si suddivida ciascun super blocco in \textit{blocchi} di dimensione $\log 2n$ (secondo livello). Si ponga per comodità $\log{2n}=k$.

\subsection{Gestione Primo Livello}
Il numero di super blocchi in questo livello è $\frac{2n}{k^2}$ (numerati da $0$ a $\frac{2n}{k^2}-1$), per ognuno di essi si calcoli l'eccesso minimo presente ottenendo la sequenza $S$:
    $$
    S_i = \min_{ik^2 \leq j < (i+1)k^2}\{t^v_j\} \qquad 0\leq i < \frac{2n}{k^2}
    $$
Si costruisca una Segment Tree relativo a questa sequenza: poiché gli elementi di $t^v$ sono compresi tra $0$ e $2n$ (poiché sono gli eccessi di una sequenza di $2n$ parentesi) e la sequenza $S$ è lunga $\frac{2n}{k^2}$, tale Segment Tree occuperà un numero di bit pari a $\mathcal{O}(\frac{2n}{k^2}\log{2n})=\mathcal{O}(\frac{2n}{\log{2n}})=o(n)$.

Inoltre si calcoli la sequenza $T$ composta dall'eccesso iniziale di ciascun super blocco:
    $$
    T_i = t^v_{ik^2} \qquad 0\leq i < \frac{2n}{k^2}
    $$
salvando la sequenza $T$ in un array il numero di bit necessari per memorizzarla sarà anch'esso $\frac{2n}{k^2}\log{2n}=o(n)$, quindi in totale il numero di bit utilizzati per salvare in memoria questo primo livello di informazioni è $o(n)$.

\subsection{Gestione Secondo Livello}
Il numero di blocchi in questo livello è $\frac{2n}{k}$ (numerati da $0$ a $\frac{2n}{k}-1$), per ognuno di essi si calcoli l'eccesso minimo presente come differenza rispetto all'eccesso iniziale del relativo super blocco:
    $$
    B_i = \left(\min_{ik \leq j < (i+1)k}\{t^v_j\}\right)-T_{\left\lfloor{\frac{i}{k}}\right\rfloor} \qquad 0\leq i < \frac{2n}{k}
    $$
salvando la sequenza $B$ in un array il numero di bit necessari per memorizzarla è $\frac{2n}{k}\log{(2(\log{2n})^2)}=\frac{2n}{\log{2n}}(2\log{\log{2n}}+\log{2})=o(n)$ perchè ciascun elemento di $B$ sarà compreso tra $-(\log{2n})^2$ e $(\log{2n})^2$.

Inoltre si calcoli la sequenza $A$ composta dalla differenza tra l'eccesso iniziale di ciascun blocco e l'eccesso del relativo super blocco:
    $$
    A_i = t^v_{ik}-T_{\left\lfloor{\frac{i}{k}}\right\rfloor} \qquad 0\leq i < \frac{2n}{k}
    $$
salvando la sequenza $A$ in un array il numero di bit necessari per memorizzarla è $\frac{2n}{k}\log{(2(\log{2n})^2)}=o(n)$, quindi in totale il numero di bit utilizzati per salvare in memoria questo secondo livello di informazioni è $o(n)$.

\section{find\_close e find\_open}
Una volta creato il Segment Tree relativo a $S$ (la cui radice verrà indicata con \texttt{root}) e calcolati i valori di $T$, $B$ e $A$ relativi alla sequenza di parentesi $v$ è possibile eseguire l'operazione \texttt{find\_close(i)} (supponendo che la parentesi in posizione $i$ sia aperta) nel seguente modo:
    \begin{enumerate}
        \item Si calcola l'eccesso $u = t^v_i = T_{\left\lfloor{\frac{i}{k^2}}\right\rfloor}+A_{\left\lfloor{\frac{i}{k}}\right\rfloor} + (t^v_i-t^v_{k\left(\left\lfloor{\frac{i}{k}}\right\rfloor\right)})= T_{\left\lfloor{\frac{i}{k^2}}\right\rfloor}+A_{\left\lfloor{\frac{i}{k}}\right\rfloor}+2\sum_{j=k(\left\lfloor{\frac{i}{k}}\right\rfloor)}^{i-1} v_j-(i-k(\left\lfloor{\frac{i}{k}}\right\rfloor))+(1-v_{k(\left\lfloor{\frac{i}{k}}\right\rfloor)})$ che è possibile calcolare in $\mathcal{O}(\log{n})$ (la sommatoria contiene al più $\log{2n}$ addendi);
        \item Si verifica se la posizione cercata appartiene allo stesso blocco dell'indice $i$ effettuando una scansione lineare di tutte le parentesi che seguono $i$ fino alla fine del blocco a cui esso appartiene ($v_{i+1},v_{i+2},\dots,v_{k(\left\lfloor{\frac{i}{k}}\right\rfloor+1)-1}$): se vi è un indice $x$ tale che il numero di parentesi aperte e chiuse comprese fra $i$ e $x$ è uguale allora tale indice $x$ è la posizione cercata. Durante questo processo viene fatto accesso al più $k=\log 2n$ volte alla sequenza $v$;
        \item Altrimenti si verifica se la posizione cercata appartiene allo stesso super blocco dell'indice $i$ effettuando una scansione dei valori di $B$ relativi ai blocchi successivi a quello di $i$ fino alla fine del super blocco a cui appartiene $i$ ($B_{\left\lfloor{\frac{i}{k}}\right\rfloor+1},\dots,B_{k(\left\lfloor{\frac{i}{k^2}}\right\rfloor+1)-1}$): se vi è un indice $x$ tale che $B_x+T_{\left\lfloor{\frac{i}{k^2}}\right\rfloor} \leq t^v_i$ allora la posizione cercata si trova nel blocco $x$ (e questo blocco si trova nello stesso super blocco di $i$), in tal caso per trovare la posizione esatta basta una scansione di tutte le parentesi di tale blocco ($v_{xk},v_{xk+1},\dots,v_{xk+k-1}$) finchè non si trova il primo indice $y$ in cui vale $t^v_y=t^v_i$ (si noti come sia possibile calcolare mano a mano il valore di $t^v_y$ utilizzando la stessa formula del punto $1$). Durante questo processo viene fatto accesso al più $k=\log 2n$ volte alla sequenza $B$ e al più $k=\log 2n$ volte alla sequenza $v$;
        \item Se non è stata trovata la posizione dopo gli step $2$ e $3$ significa che la posizione cercata si trova in un altro super blocco rispetto a quello in cui si trova $i$, per trovare tale super blocco basterà chiamare la procedura \texttt{Find\_succ}$(B,\left\lfloor{\frac{i}{k^2}}\right\rfloor,t^v_i)$ che provvederà a ritornare l'indice $x$ del primo super blocco che segue $i$ contenente un eccesso minore o uguale a $t^v_i$ (e poichè due valori consecutivi $t^v$ differiscono di al più $1$, quel super blocco conterrà sicuramente un eccesso uguale a $t^v_i$). A quel punto va ricercato il blocco in cui si trova la posizione cercata e dopodiché si ricerca la posizione esatta in modo analogo a quanto spiegato nello step $3$: si effettua una scansione dei valori di $B$ relativi ai blocchi contenuti nel super blocco $x$ ($B_{xk},\dots,B_{xk+k-1}$) e se vi è un indice $y$ tale che $B_y+T_x \leq t^v_i$ allora la posizione cercata si trova nel blocco $y$, infine per trovare la posizione esatta basta una scansione di tutte le parentesi di tale blocco ($v_{xk^2+yk},\dots,v_{xk^2+yk+k-1}$) finchè non si trova il primo indice $z$ in cui vale $t^v_z=t^v_i$. Durante questo processo viene effettuata una chiamata alla funzione \texttt{Find\_succ} (che ha complessità $\mathcal{O}(\log{\frac{2n}{k^2}})=\mathcal{O}(\log{n})$) e viene fatto accesso al più $k=\log 2n$ volta alla sequenza $B$ e al più $k=\log 2n$ volta alla sequenza $v$.
    \end{enumerate}
In conclusione è possibile effettuare l'operazione \texttt{find\_close} con complessità $\mathcal{O}(\log{n})$ utilizzando la sequenza $v$ di parentesi che occupa $2n$ bit e altre strutture ausiliare (il Segment Tree relativo alla sequenza $S$ e le sequenze $T,B,A$) che occupano $o(n)$ bit. Il seguente pseudocodice mostra l'implementazione di tutta la procedura:
    \begin{algorithm}
    \caption{\texttt{Find\_close}}\label{findclose}
    \begin{algorithmic}[1]
    \Procedure{Find\_close}{$root,T,A,B,n,v,i$}\Comment{si suppone $v_i$ parentesi aperta}
        \State $k\gets\log{2n}$
        \State $u\gets T_{\left\lfloor{\frac{i}{k^2}}\right\rfloor}+A_{\left\lfloor{\frac{i}{k}}\right\rfloor}-(i-k(\left\lfloor{\frac{i}{k}}\right\rfloor))$\Comment{step 1}
        \For{$j\gets k(\left\lfloor{\frac{i}{k}}\right\rfloor)\ \textbf{to}\ i-1$}
            \State $u\gets u+2v_j$
        \EndFor
        \State $u\gets u+(1-v_{k(\left\lfloor{\frac{i}{k}}\right\rfloor)})$
        \State
        \State $tmp\gets 0$\Comment{step 2}
        \For{$x\gets i\ \textbf{to}\ k(\left\lfloor{\frac{i}{k}}\right\rfloor+1)-1$}
            \If{$v_x=1$}
                \State $tmp\gets tmp+1$
            \Else  
                \State $tmp\gets tmp-1$
            \EndIf
            \If{$tmp=0$}
                \State \textbf{return} $x$
            \EndIf
        \EndFor
        \State
        \For{$x\gets \left\lfloor{\frac{i}{k}}\right\rfloor+1\ \textbf{to}\ k(\left\lfloor{\frac{i}{k^2}}\right\rfloor+1)-1$}\Comment{step 3}
            \If{$B_x+T_{\left\lfloor{\frac{i}{k^2}}\right\rfloor} \leq u$}
                \State $currt\gets T_{\left\lfloor{\frac{i}{k^2}}\right\rfloor}+A_x$
                \For{$y\gets xk\ \textbf{to}\ xk+k-1$}
                    \If{$currt=u$}
                        \State \textbf{return} $y$
                    \EndIf
    \algstore{alg1}            
    \end{algorithmic}
    \end{algorithm}
    \begin{algorithm}
    \begin{algorithmic}[1]
    \algrestore{alg1}
                    \If{$v_y=1$}
                        \State $currt\gets currt+1$
                    \EndIf
                    \If{$v_{y+1}=0$}
                        \State $currt\gets currt-1$
                    \EndIf
                \EndFor
            \EndIf
        \EndFor
        \State
        \State $x\gets \Call{Find\_succ}{root,B,\left\lfloor{\frac{i}{k^2}}\right\rfloor,u}$ \Comment{step 4}
        \For{$y\gets xk\ \textbf{to}\ xk+k-1$}
            \If{$B_y+T_x \leq u$}
                \State $currt\gets T_x+A_y$
                \For{$z\gets xk^2+yk\ \textbf{to}\ xk^2+yk+k-1$}
                    \If{$currt=u$}
                        \State \textbf{return} $z$
                    \EndIf
                    \If{$v_z=1$}
                        \State $currt\gets currt+1$
                    \EndIf
                    \If{$v_{z+1}=0$}
                        \State $currt\gets currt-1$
                    \EndIf
                \EndFor
            \EndIf
        \EndFor
    \EndProcedure
    \end{algorithmic}
    \end{algorithm}
    
In modo simmetrico è possibile implementare l'operazione \texttt{find\_open(i)}: basti notare che l'operazione \texttt{find\_open(i)} relativa ad una sequenza $v=v_0,\dots,v_{2n-1}$ è equivalente a un'operazione di \texttt{find\_close(2n-1-i)} relativa alla sequenza sequenza $v_{2n-1},\dots,v_0$. Inizialmente si verifica se la posizione cercata è presente nello stesso blocco di $i$, altrimenti si verifica se è presente nello stesso super blocco di $i$ e in caso contrario si va a ricercare il super blocco di appartenenza, il blocco corretto all'interno del super blocco e infine la posizione esatta all'interno del blocco:
    \begin{algorithm}[H]
    \caption{\texttt{Find\_open}}\label{findopen}
    \begin{algorithmic}[1]
    \Procedure{Find\_open}{$root,T,A,B,n,v,i$}\Comment{si suppone $v_i$ parentesi chiusa}
        \State $k\gets\log{2n}$
        \State $u\gets T_{\left\lfloor{\frac{i}{k^2}}\right\rfloor}+A_{\left\lfloor{\frac{i}{k}}\right\rfloor}-(i-k(\left\lfloor{\frac{i}{k}}\right\rfloor)+1)$\Comment{step 1}
        \For{$j\gets k(\left\lfloor{\frac{i}{k}}\right\rfloor)\ \textbf{to}\ i-1$}
            \State $u\gets u+2v_j$
        \EndFor
        \State $u\gets u+(1-v_{k(\left\lfloor{\frac{i}{k}}\right\rfloor)})$
        \State
        \State $tmp\gets 0$\Comment{step 2}
        \For{$x\gets i\ \textbf{down to}\ k\left\lfloor{\frac{i}{k}}\right\rfloor$}
            \If{$v_x=1$}
                \State $tmp\gets tmp+1$
            \Else  
                \State $tmp\gets tmp-1$
            \EndIf
            \If{$tmp=0$}
                \State \textbf{return} $x$
            \EndIf
        \EndFor
    \algstore{alg1}            
    \end{algorithmic}
    \end{algorithm}
    \begin{algorithm}[ht]
    \begin{algorithmic}[1]
    \algrestore{alg1}
        \For{$x\gets \left\lfloor{\frac{i}{k}}\right\rfloor-1\ \textbf{down to}\ k(\left\lfloor{\frac{i}{k^2}}\right\rfloor)$}\Comment{step 3}
            \If{$B_x+T_{\left\lfloor{\frac{i}{k^2}}\right\rfloor} \leq u$}
                \State $currt\gets T_{\left\lfloor{\frac{i}{k^2}}\right\rfloor}+A_{x+1}$
                \For{$y\gets xk+k-1\ \textbf{down to}\ xk$}
                    \If{$v_{y+1}=0$}
                        \State $currt\gets currt+1$
                    \EndIf
                    \If{$v_y=1$}
                        \State $currt\gets currt-1$
                    \EndIf
                    \If{$currt=u$}
                        \State \textbf{return} $y$
                    \EndIf
                \EndFor
            \EndIf
        \EndFor
        \State
        \State $x\gets \Call{Find\_prev}{root,B,\left\lfloor{\frac{i}{k^2}}\right\rfloor,u}$ \Comment{step 4}
        \For{$y\gets xk+k-1\ \textbf{down to}\ xk$}
            \If{$B_y+T_x \leq u$}
                \If{$y<xk+k-1$}
                    \State $currt\gets T_x+A_{y+1}$
                \Else
                    \State $currt\gets T_{x+1}$
                \EndIf
                \For{$z\gets xk^2+yk+k-1\ \textbf{down to}\ xk^2+yk$}
                    \If{$v_{z+1}=0$}
                        \State $currt\gets currt+1$
                    \EndIf
                    \If{$v_z=1$}
                        \State $currt\gets currt-1$
                    \EndIf
                    \If{$currt=u$}
                        \State \textbf{return} $z$
                    \EndIf
                \EndFor
            \EndIf
        \EndFor
    \EndProcedure
    \end{algorithmic}
    \end{algorithm}
 
\section{Find\_enclose}
L'operazione \texttt{left\_enclose(i)} è analoga a \texttt{find\_open(i)} con l'unica differenza che l'eccesso cercato non è pari a $t^v_i$ ma bensì $t^v_i-1$ (e che non esiste una parentesi con tale eccesso se $t^v_i=0$):
    \begin{algorithm}[H]
    \caption{\texttt{Left\_enclose}}\label{leftenclose}
    \begin{algorithmic}[1]
    \Procedure{Left\_enclose}{$root,T,A,B,n,v,i$}
        \State $k\gets\log{2n}$
        \State $u\gets T_{\left\lfloor{\frac{i}{k^2}}\right\rfloor}+A_{\left\lfloor{\frac{i}{k}}\right\rfloor}-(i-k(\left\lfloor{\frac{i}{k}}\right\rfloor))$\Comment{step 1}
        \For{$j\gets k(\left\lfloor{\frac{i}{k}}\right\rfloor)\ \textbf{to}\ i-1$}
            \State $u\gets u+2v_j$
        \EndFor
        \State $u\gets u+(1-v_{k(\left\lfloor{\frac{i}{k}}\right\rfloor)})$
        \If{$v_i=0$}
            \State $u\gets u-1$
        \EndIf
        \State $u\gets u-1$\Comment{l'eccesso cercato ora è $t^v_i-1$}
    \algstore{alg1}            
    \end{algorithmic}
    \end{algorithm}    
    \begin{algorithm}[H]
    \begin{algorithmic}[1]
    \algrestore{alg1}
        \If{u=-1}
            \State \textbf{return} $-1$ \Comment{Non esiste una coppia di parentesi che racchiude l'indice $i$}
        \EndIf
        \State
        \State $tmp\gets u+1$\Comment{step 2}
        \For{$x\gets i\ \textbf{down to}\ k\left\lfloor{\frac{i}{k}}\right\rfloor$}
            \If{$tmp=u$}
                \State \textbf{return} $x$
            \EndIf
            \If{$v_x=0$}
                \State $tmp\gets tmp+1$
            \EndIf
            \If{$v_{x-1}=1$}
                \State $tmp\gets tmp-1$
            \EndIf
        \EndFor
        \State
        \For{$x\gets \left\lfloor{\frac{i}{k}}\right\rfloor-1\ \textbf{down to}\ k(\left\lfloor{\frac{i}{k^2}}\right\rfloor)$}\Comment{step 3}
            \If{$B_x+T_{\left\lfloor{\frac{i}{k^2}}\right\rfloor} \leq u$}
                \State $currt\gets T_{\left\lfloor{\frac{i}{k^2}}\right\rfloor}+A_{x+1}$
                \For{$y\gets xk+k-1\ \textbf{down to}\ xk$}
                    \If{$v_{y+1}=0$}
                        \State $currt\gets currt+1$
                    \EndIf
                    \If{$v_y=1$}
                        \State $currt\gets currt-1$
                    \EndIf
                    \If{$currt=u$}
                        \State \textbf{return} $y$
                    \EndIf
                \EndFor
            \EndIf
        \EndFor
        \State
        \State $x\gets \Call{Find\_prev}{root,B,\left\lfloor{\frac{i}{k^2}}\right\rfloor,u}$ \Comment{step 4}
        \For{$y\gets xk+k-1\ \textbf{down to}\ xk$}
            \If{$B_y+T_x \leq u$}
                \If{$y<xk+k-1$}
                    \State $currt\gets T_x+A_{y+1}$
                \Else
                    \State $currt\gets T_{x+1}$
                \EndIf
                \For{$z\gets xk^2+yk+k-1\ \textbf{down to}\ xk^2+yk$}
                    \If{$v_{z+1}=0$}
                        \State $currt\gets currt+1$
                    \EndIf
                    \If{$v_z=1$}
                        \State $currt\gets currt-1$
                    \EndIf
                    \If{$currt=u$}
                        \State \textbf{return} $z$
                    \EndIf
                \EndFor
            \EndIf
        \EndFor
    \EndProcedure
    \end{algorithmic}
    \end{algorithm}
E di conseguenza l'operazione \texttt{find\_enclose} non è nient'altro che una chiamata a \texttt{left\_enclose} e \texttt{find\_close}:
    \begin{algorithm}[H]
    \caption{\texttt{Find\_enclose}}\label{findenclose}
    \begin{algorithmic}[1]
    \Procedure{Find\_enclose}{$root,T,A,B,n,v,i$}
    \State $x\gets$ \Call{Left\_enclose}{$root,T,A,B,n,v,i$}
    \If{$x=-1$}
        \State \textbf{return} $(-1,-1)$ \Comment{non esiste soluzione}
    \Else
        \State \textbf{return} $(x,$\Call{Find\_close}{$x$}$)$
    \EndIf
    \EndProcedure
    \end{algorithmic}
    \end{algorithm}
\section{Conclusioni}
È stato mostrato come sia possibile effettuare le operazioni \texttt{find\_close}, \texttt{find\_open} e \texttt{find\_enclose} con complessità $\mathcal{O}(\log{n})$ utilizzando $2n+o(n)$ bits. Inoltre la costruzione iniziale di tutte le strutture necessarie ha complessità $\mathcal{O}(n)$.
\end{document}
